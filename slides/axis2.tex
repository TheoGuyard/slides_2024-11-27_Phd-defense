\section{Axis 2 -- How to solve relaxations efficiently ?}

\begin{frame}{Axis 2 -- Convex optimization}
  \begin{tikzpicture}[remember picture,overlay]
    \begin{scope}[xshift=0.5\linewidth]
      \onslide<1-> {
        \node[text width=0.4\linewidth, align=center] (relax) at (0,3.25) {
          \begin{blockcolor}{mDarkTeal}{Relaxation for region $\nodeSymb$}
            \centering
            $\min_{\pv \in \nodeSymb} \lfunc(\dic\pv) + \rfunc_{\text{cvx}}(\pv)$
          \end{blockcolor}
        };
      }
      %
      %
      %
      \onslide<2-> {
        \node[text width=0.4\linewidth, align=center] (problem) at ($(relax)+(0,-2.5)$) {
          \begin{blockcolor}{mDarkTeal}{Convex problem}
              \centering
              $\min_{\textcolor{mLightBrown}{\pv \in \kR^{\pdim}}} \lfunc(\dic\pv) + \textcolor{mLightBrown}{\relaxrfunc}(\pv)$
          \end{blockcolor}
        };
        %
        \draw[ultra thick,<->] ($(relax.south)+(0,0.1)$) -- ($(problem.north)+(0,-0.3)$) node[midway,draw,ultra thick,fill=mLightWhite] {reformulation};
      }
      %
      %
      %
      \onslide<3-> {
          \node[text width=0.8\linewidth,align=left,anchor=west,font=\small] at ($(problem.east)+(0,-0.1)$) {$\relaxrfunc$ proper, closed, convex \\ $\relaxrfunc$ separable \\ $\relaxrfunc$ non-smooth at $\pv=\0$};
      }
      %
      %
      %
      \onslide<4-> {
          \node[text width=0.8\linewidth,align=left,anchor=west,font=\small] at ($(relax.east)+(0,-0.1)$) {remind $\rfunc_{\text{cvx}}$ acts as \\  $\ell_1$-norm near $\pv = \0$};
      }
      %
      %
      %
      \onslide<5-> {
        \node[text width=0.4\linewidth,align=center] (beyond) at ($(problem)+(-4.25,0.8)$) {\textbf{Applications beyond \\ the BnB scope}};
        \draw[ultra thick,->] ($(beyond.south)+(0,0)$) .. controls ($(beyond.south)+(0,-0.4)$) .. ($(problem.west)+(0,-0.1)$);
      }
      %
      %
      %
      \onslide<6-> {
        \node[text width=0.4\linewidth,align=center,anchor=north] (method) at ($(problem)+(-3,-1.25)$) {\textbf{$\1^{\text{st}}$-order methods} \\ Proximal gradient \\ Coordinate descent \\ ...};
        %
        \node (x0) at ($(method.south)+(-2,-0.8)$) {$\iter{\pv}{0}$};
        \node[draw,ultra thick] (xk) at ($(method.south)+(0,-0.8)$) {$\iter{\pv}{k} \rightarrow \iter{\pv}{k+1}$};
        \node[yshift=6,font=\small] at (xk.north) {iteration};
        \node (xopt) at ($(method.south)+(2,-0.8)$) {};
        %
        \draw[ultra thick] (x0) -- (xk);
        \draw[ultra thick] (xk) -- ($(xk)!0.65!(xopt)$) -- ($(xk)!0.65!(xopt)+(0,-0.65)$) -- ($(x0)!0.35!(xk)+(0,-0.65)$) -- ($(x0)!0.35!(xk)$);
        \draw[ultra thick,->] ($(xk)!0.65!(xopt)+(0,-0.65)$) -- ($(xk)+(-0.2,-0.65)$);
        \node[right,yshift=1] at (xopt) {$\opt{\pv}$};
        \draw[ultra thick,->] (xk) -- (xopt);
      }
      %
      %
      %
      \onslide<7-> {
        \node[text width=0.5\linewidth,align=center,anchor=north] (cost) at ($(problem)+(3,-1.25)$) {\textbf{Solving cost}};
        \node[text width=0.5\linewidth,align=center] (cost-formula) at ($(cost)+(0,-1)$) {cost per iteration \\ $\times$ \\ number of iterations};
      }
      %
      %
      %
      \onslide<8-> {
        \node[mLightBrown] at ($(cost-formula.south)+(-1.75,-0.75)$) {dimension of $\pv$};
        \draw[ultra thick,<-,mLightBrown] ($(cost-formula.south)+(-1.75,-0.4)$) .. controls ($(cost-formula.south)+(-1.75,1.2)$) .. ($(cost-formula.south)+(-1.5,1.2)$);
      }
      %
      %
      %
      \onslide<9-> {
        \node[mLightBrown] at ($(cost-formula.south)+(1.75,-0.75)$) {regularity of $\lfunc/\relaxrfunc$};
        \draw[ultra thick,<-,mLightBrown] ($(cost-formula.south)+(1.75,-0.4)$) .. controls ($(cost-formula.south)+(1.75,0.2)$) .. ($(cost-formula.south)+(1.6,0.2)$);
      }
    \end{scope}
  \end{tikzpicture}
\end{frame}

\begin{frame}{Axis 2 -- Sparse structure exploitation}
  \begin{tikzpicture}[remember picture,overlay]
    \begin{scope}[xshift=0.5\linewidth]
      \onslide<1-> {
        \node[text width=0.35\linewidth, align=center] (problem) at (0,3.25) {
          \begin{blockcolor}{mDarkTeal}{Convex problem}
            \centering
            $\min_{\pv \in \kR^{\pdim}} \lfunc(\dic\pv) + \relaxrfunc(\pv)$
          \end{blockcolor}
        };
        %
        \node at ($(problem)+(0,-1.5)
        $) (question) {\textbf{Task}};
        \node[draw,ultra thick,text width=0.33\linewidth, align=center] at ($(question.south)+(0,-0.25)$) (question1-text) {Reduce the solving cost};
      }
      %
      %
      %
      \onslide<2-> {
        \node[text width=0.4\linewidth,align=center] (opt1) at ($(question)+(-3.5,-2)$) {\textbf{Option 1} \\ Reduce the dimension of $\pv$};
        \draw[->,ultra thick] ($(question1-text.west)+(0.05,0)$) .. controls ($(opt1)+(0,1)$) .. (opt1);
      }
      %
      %
      %
      \onslide<3-> {
        \node[font=\small] at ($(problem)+(4,0.3)
        $) (property) {$\relaxrfunc$ non-smooth at $\pv=\0$};
        \node[font=\small] at ($(property)+(0,-0.8)
        $) (property-text) {sparse solution $\opt{\pv}$};
        \draw[ultra thick,->] (property) -- (property-text);
      }
      %
      %
      %
      \onslide<4-> {
        \node[text width=\linewidth,align=center,anchor=north] (scr) at ($(opt1)+(0,-1)$) {\textbf{Screening tests} \\ Identify zeros in $\opt{\pv}$ \\ \small{L. El Ghaoui \textit{et al.} (2011)} \\ E. Ndiaye \textit{et al.} (2020)};
        \draw[ultra thick,->] (opt1) -- (scr);
      }
      %
      %
      %
      \onslide<5-> {
        \node[text width=0.4\linewidth,align=center] (opt2) at ($(question)+(3.5,-2)$) {\textbf{Option 2} \\ Improve the regularity of $\lfunc/\relaxrfunc$};
        \draw[->,ultra thick] ($(question1-text.east)+(-0.05,0)$) .. controls ($(opt2)+(0,1)$) .. (opt2);
      }
      %
      %
      %
      \onslide<6-> {
        \node[text width=\linewidth,align=center,anchor=north,mLightBrown] (smt) at ($(opt2)+(0,-1)$) {\textbf{Smoothing tests} \\ Identify non-zeros in $\opt{\pv}$};
        \draw[ultra thick,->] (opt2) -- (smt);
      }
      %
      %
      %
      \onslide<7-> {
        \draw[<->,ultra thick,dashed] ($(scr.north)+(2,-0.25)$) -- ($(smt.north)+(-2,-0.25)$);
      }
    \end{scope}
  \end{tikzpicture}
\end{frame}

\begin{frame}{Axis 2 -- Screening and smoothing tests}
  \begin{tikzpicture}[remember picture,overlay]
    \begin{scope}[xshift=0.5\textwidth]
      \onslide<1-> {
        \node[text width=0.325\linewidth, align=center] (problem) at (-3.5,3.25) {
          \begin{blockcolor}{mDarkTeal}{Convex problem}
            \centering
            $\min_{\pv \in \kR^{\pdim}} \lfunc(\dic\pv) + \relaxrfunc(\pv)$
          \end{blockcolor}
        };
      }
      %
      %
      %
      \onslide<2-> {
        \node[align=center,text width=0.325\textwidth] (dual) at (3.5,3.25) {
          \begin{blockcolor}{mDarkTeal}{Dual problem}
              \centering
              $\max_{\dv \in \kR^{\ddim}} \dfunc(\dv)$
          \end{blockcolor}
        };
      }
      %
      %
      %
      \onslide<3-> {
        \node[align=center,text width=0.3\textwidth] (optcond) at (0,1.5) {
          \begin{blockcolor}{mDarkTeal}{Opt. conditions}
              \centering
              $\transp{\dic}\opt{\dv} \in \subdiff\relaxrfunc(\opt{\pv})$
          \end{blockcolor}
        };
        %
        \draw[ultra thick,->] (problem) .. controls ($(problem)+(0,-1.75)$) .. (optcond) node[midway,fill=mLightWhite,draw,ultra thick,font=\small] {solution $\opt{\pv}$};
        %
        \draw[ultra thick,->] (dual) .. controls ($(dual)+(0,-1.75)$) .. (optcond) node[midway,fill=mLightWhite,draw,ultra thick,font=\small] {solution $\opt{\dv}$};
        %
        \node[font=\scriptsize] at ($(optcond.north)+(0,-0.15)$) {if strong duality holds};
      }
      %
      %
      %
      \onslide<4-> {
        \node[font=\scriptsize] at ($(optcond.south)+(0,-0.15)$) {remind $\relaxrfunc(\pv)=\sum_{\idxentry=1}^{\pdim}\separable{\relaxrfunc}{\idxentry}(\pvi{\idxentry})$};
      }
      %
      %
      %
      \onslide<5-> {
        \node (safe) at ($(dual)+(-0.15,-2.9)$) {\textbf{Safe region}};
        \node at ($(safe.south)+(0,-0.15)$) (result-text) {$\saferegion \subseteq \kR^{\ddim}$ with $\opt{\dv} \in \saferegion$};
        %
        \draw[ultra thick,<->] (safe) -- ($(safe)+(0,0.8)$);
      }
      %
      %
      %
      \onslide<6-> {
        \node[mLightBrown] (spotlight) at (0,-0.75) {\textbf{Spotlight result}};
        \node[draw,ultra thick,mLightBrown] at ($(spotlight.south)+(0,-0.25)$) (result-text) {Some zeros and non-zeros in $\opt{\pv}$ can be identified from a safe region $\saferegion$.};
        %
        \node[text width=\linewidth,align=left] (theorem) at (0,-2.75) {
          \textbf{Theorem --} Given a safe region $\saferegion$, note $\transp{\dici{}}\saferegion = \kset{\transp{\dici{}}\dv}{\dv \in \saferegion}$, one has
          \begin{alignat*}{6}
            \text{Screening test:} &&\quad \transp{\dici{\idxentry}}\saferegion &\subseteq \interior(\subdiff\separable{\relaxrfunc}{\idxentry}(0)) &&\implies \opt{\pvi{\idxentry}} = 0 \\
            \text{Smoothing test:} &&\quad \transp{\dici{\idxentry}}\saferegion &\subseteq \complset(\subdiff\separable{\relaxrfunc}{\idxentry}(0)) &&\implies \opt{\pvi{\idxentry}} \neq 0
          \end{alignat*}
        };
      }
      %
      %
      %
      \onslide<7-> {
        \draw[ultra thick,mLightBrown] ($(theorem)+(-1.3,-0.8)$) -- ($(theorem)+(-1.3,-1)$) -- ($(theorem)+(2,-1)$) -- ($(theorem)+(2,-0.8)$);
        \node[mLightBrown,font=\small] at (0.4,-4) (easy) {easy to evaluate if $\saferegion$ has a simple shape};
      }
    \end{scope}
  \end{tikzpicture}
\end{frame}

\begin{frame}{Axis 2 -- Dynamic identification}
  \begin{tikzpicture}[remember picture,overlay]
    \begin{scope}[xshift=0.5\linewidth]
      \onslide<1-> {
        \node[text width=0.325\linewidth, align=center] (problem) at (0,3.25) {
          \begin{blockcolor}{mDarkTeal}{Convex problem}
              \centering
              $\min_{\pv \in \kR^{\pdim}} \lfunc(\dic\pv) + \relaxrfunc(\pv)$
          \end{blockcolor}
        };
        %
        \node (x0) at (-2.75,1.5) {$\iter{\pv}{0}$};
        \node[draw,ultra thick] (xk) at (0,1.5) {$\iter{\pv}{k} \rightarrow \iter{\pv}{k+1}$};
        \node[yshift=6,font=\small] at (xk.north) {iteration};
        \node (xopt) at (2.75,1.5) {};
        %
        \draw[ultra thick] (x0) -- (xk);
        \draw[ultra thick] (xk) -- ($(xk)!0.55!(xopt)$) -- ($(xk)!0.55!(xopt)+(0,-0.65)$) -- ($(x0)!0.45!(xk)+(0,-0.65)$) -- ($(x0)!0.45!(xk)$);
        \draw[ultra thick,->] ($(xk)!0.55!(xopt)+(0,-0.65)$) -- ($(xk)+(-0.2,-0.65)$);
        \node[right] at (xopt) {$\opt{\pv}$};
        \draw[ultra thick,->] (xk) -- (xopt);
        %
        \node[text width=0.5\linewidth,align=center,anchor=north,font=\small] (cost) at ($(problem)+(4,0.35)$) {\textbf{Solving cost}};
        \node[text width=0.5\linewidth,align=center,font=\small] (cost-formula) at ($(cost)+(0,-0.5)$) {iter. cost $\times$ number of iter.};
        %
        % \node[mLightBrown] at ($(cost-formula.south)+(-1.75,-0.75)$) {dimension of $\pv$};
        % \draw[ultra thick,<-,mLightBrown] ($(cost-formula.south)+(-1.75,-0.4)$) .. controls ($(cost-formula.south)+(-1.75,1.2)$) .. ($(cost-formula.south)+(-1.5,1.2)$);
        %
        % \node[mLightBrown] at ($(cost-formula.south)+(1.75,-0.75)$) {regularity of $\lfunc/\relaxrfunc$};
        % \draw[ultra thick,<-,mLightBrown] ($(cost-formula.south)+(1.75,-0.4)$) .. controls ($(cost-formula.south)+(1.75,0.2)$) .. ($(cost-formula.south)+(1.6,0.2)$);
      }
      %
      %
      %
      \onslide<2-> {
        \fill[mLightWhite] ($(xk)!0.55!(xopt)+(0,-0.6)$) rectangle ($(x0)!0.45!(xk)+(0,-0.7)$);
        \fill[mLightWhite] ($(xk)+(-0.5,-0.5)$) rectangle ($(xk)+(0.5,-0.8)$);
        %
        \draw[ultra thick] (xk) -- ($(xk)!0.55!(xopt)$) -- ($(xk)!0.55!(xopt)+(0,-2)$) -- ($(x0)!0.45!(xk)+(0,-2)$) -- ($(x0)!0.45!(xk)$);
        %
        \draw[ultra thick,->] ($(xk)!0.55!(xopt)$) -- ($(xk)!0.55!(xopt)+(0,-0.5)$);
        \draw[ultra thick,->] ($(xk)!0.55!(xopt)$) -- ($(xk)!0.55!(xopt)+(0,-1.75)$);
        %
        \draw[ultra thick,->] ($(x0)!0.45!(xk)+(0,-2)$) -- ($(x0)!0.45!(xk)+(0,-0.25)$);
        \draw[ultra thick,->] ($(x0)!0.45!(xk)+(0,-2)$) -- ($(x0)!0.45!(xk)+(0,-1.5)$);
        %
        \draw[ultra thick,->] ($(xk)!0.55!(xopt)+(0,-2)$) -- ($(xk)+(-0.2,-2)$);
      }
      %
      %
      %
      \onslide<3-> {
        \node[draw,ultra thick,fill=mLightWhite,font=\small,text width=0.2\linewidth,align=center] (safe) at ($(xk)!0.55!(xopt)+(0,-1)$) {new safe region};
      }
      %
      %
      %
      \onslide<4-> {
        \node[draw,ultra thick,fill=mLightWhite,font=\small,text width=0.2\linewidth,align=center] (scr) at ($(xk)+(0,-2)$) {screening test};
        %
        \node[font=\small] (identif-zero) at ($(scr)+(0,-0.75)$) {Identify some $\opt{\pvi{\idxentry}} = 0$};
      }
      %
      %
      %
      \onslide<5-> {
        \node[font=\small,anchor=north,text width=0.3\linewidth,align=center] (action-zero) at ($(identif-zero)+(0,-0.75)$) {Set $\pvi{\idxentry} = 0$ \\ \textcolor{mLightBrown}{Reduce iter. cost}};
        \draw[ultra thick,->] (identif-zero) -- (action-zero);
      }
      %
      %
      %
      \onslide<6-> {
        \node[draw,ultra thick,mLightBrown,fill=mLightWhite,font=\small,text width=0.2\linewidth,align=center] (smt) at ($(x0)!0.45!(xk)+(0,-1)$) {smoothing test};
        %
        \node[font=\small] (identif-nzero) at ($(smt)+(-2.9,0)$) {Identify some $\opt{\pvi{\idxentry}} \neq 0$};
      }
      %
      %
      %
      \onslide<7-> {
        \node[font=\small,text width=0.3\linewidth,align=center,anchor=north] (action-nzero) at ($(identif-nzero)+(0,-0.75)$) {Smooth $\separable{\relaxrfunc}{\idxentry}$, use $2^{\text{nd}}$-order iterations \\ \textcolor{mLightBrown}{Reduce number of iter.}};
        \draw[ultra thick,->] (identif-nzero) -- (action-nzero);
      }
      %
      %
      %
      \onslide<8-> {
        \node[font=\small] (enet) at ($(action-nzero)+(0,-1.25)$) {$\separable{\relaxrfunc}{\idxentry}(\pvi{}) = \abs{\pvi{}} + \pvi{}^2$};
        %
        \node[font=\small] (enet-smooth) at ($(enet)+(0,-1.5)$) {$\separable{\relaxrfunc}{\idxentry}(\pvi{}) = \pvi{} + \pvi{}^2$};
        %
        \draw[ultra thick,->] (enet) -- (enet-smooth) node[midway,fill=mLightWhite,draw,ultra thick,font=\small,inner sep=2pt] {$\opt{\pvi{\idxentry}} > 0$};
      }
      %
      %
      %
      % \onslide<8-> {
      %   \begin{scope}[xshift=-0.41\textwidth,yshift=-0.3\textheight]
      %     \draw[ultra thick,->] (-1.25,0) -- (1.25,0) node[right,font=\small] {$\pvi{}$};
      %     \draw[ultra thick,->] (0,-0.25) -- (0,1.5);
      %     \node[below left] at (0,0) {0};
      %     %
      %     \node[font=\small,gray] at (-1.25,1.3) {$\separable{\relaxrfunc}{\idxentry}$};
      %     \draw[domain=-1.25:0,smooth,variable=\x,very thick,gray] plot ({\x}, {-0.5*\x + 0.3*\x*\x});
      %   \end{scope}
      % }
      %
      %
      %
      % \onslide<8> {
      %   \begin{scope}[xshift=-0.41\textwidth,yshift=-0.3\textheight]
      %     \draw[domain=0:1.25,smooth,variable=\x,very thick,LavenderBlush4] plot ({\x}, {0.5*\x + 0.3*\x*\x});
      %   \end{scope}
      % }
      %
      %
      %
      % \onslide<9-> {
      %   \begin{scope}[xshift=-0.41\textwidth,yshift=-0.3\textheight]
      %     \draw[domain=0:1.25,smooth,variable=\x,very thick,LavenderBlush4,dashed] plot ({\x}, {0.5*\x + 0.3*\x*\x});
      %     \draw[domain=0:1,smooth,variable=\x,very thick,LavenderBlush4] plot ({\x}, {-0.5*\x + 0.3*\x*\x});
      %   \end{scope}
      % }
      %
      %
      %
      % \onslide<11-> {
      %   \node[anchor=north,font=\small] (cost) at (2,-3) {\textbf{Solving cost}};
      %   \node[font=\small] (cost-formula) at ($(cost)+(0,-0.35)$) {number of iterations $\times$ cost per iteration};
      %   \node[font=\small] at ($(cost-formula)+(-1.5,-0.35)$) {regularity of $\lfunc/\relaxrfunc$};
      %   \node[font=\small] at ($(cost-formula)+(1.5,-0.35)$) {dimension of $\pv$};
      % }
      %
      %
      %
      \onslide<9-> {
        \node[font=\small,text width=0.3\linewidth,align=center] (safe-action) at ($(safe)+(2.5,-2)$) {\textcolor{mLightBrown}{All} zeros and non-zeros identified after a finite number of iterations};
        \draw[ultra thick,->,dashed] ($(safe-action)+(-1.25,2)$) .. controls ($(safe-action)+(0,2)$) .. (safe-action) node[midway,font=\small,xshift=12,yshift=12,rotate=-45,text width=0.2\linewidth,align=center] {additional \\ conditions};
      }
      %
      %
      %
      \onslide<10-> {
        \draw[dashed,ultra thick,->,text width=0.3\linewidth,align=center] ($(safe-action)+(0,-0.75)$) -- ($(safe-action)+(0,-1.25)$) node[below,font=\small] {Smooth problem of reduced dimension};
      }
    \end{scope}
  \end{tikzpicture}
\end{frame}

\begin{frame}{Axis 2 -- Numerics}
  \begin{tikzpicture}[remember picture,overlay]
    \begin{scope}[xshift=0.5\linewidth]
      \onslide<1-> {
        \node[text width=0.35\linewidth, align=center] (problem) at (0,2.75) {
          \begin{blockcolor}{mDarkTeal}{Convex problem}
              \centering
              $\min_{\pv \in \kR^{\pdim}} \lfunc(\dic\pv) + \relaxrfunc(\pv)$
          \end{blockcolor}
        };
        %
        \node[font=\small] at ($(problem)+(0,-1)$) {$\dic \in \kR^{500\times1000}$ / $\lfunc$ : Quadratic / $\relaxrfunc$ : $\ell_1\ell_2$-norm};
      }
      %
      %
      %
      \onslide<7-> {
        \node[text width=0.75\linewidth,align=center] at (3,-2.75) {\textbf{Transform 1${}^{\text{st}}$-order \\ into 2${}^{\text{nd}}$-order iterations} \\ More costly but way less iterations \\ Gains in terms of solving time};
      }
    \end{scope}
    \begin{scope}[xshift=0.1\linewidth,yshift=-0.2\textheight]
      \onslide<2-> {
        \node[font=\small,text width=0.5\linewidth,align=left] at (7.5,1.5) {\textbf{Accel. proximal gradient} \\ \ref{plot:ProximalGradient} Vanilla method \\ \ref{plot:ProximalGradientScr} With screening \\ \ref{plot:ProximalGradientScrRlx} With screening and smoothing};
      }
      %
      %
      %
      \onslide<4-> {
        \node[font=\small,rotate=-8] at (3,1.75) {linear};
        \node[font=\small,rotate=-10] at (3,1.4) {linear};
      }
      %
      %
      %
      \onslide<6-> {
        \node[font=\small,rotate=-80] at (1.1,0.9) {super-linear};
      }
      %
      %
      %
      \onslide<2> {
        \pgfplotscreateplotcyclelist{cycle_perfprofiles_synthetic_logistic}{
          TolLightGreen, very thick, smooth\\    
          TolDarkBrown, very thick, smooth\\
          TolLightRed, very thick, smooth\\
        }
        \begin{axis}[
            width   = 0.5\textwidth,
            height  = 0.4\textwidth,,
            legend style={
                at={(1.1,0.5)},
                anchor=west,
                legend columns=1,
                draw=none
            },
            cycle list name=cycle_perfprofiles_synthetic_logistic,
            mbaseplot,
            axis line style = ultra thick,
            major tick style = {ultra thick,color=mDarkTeal},
            xmajorgrids=true,
            ymajorgrids=true,
            major grid style={dotted},
            axis x line=bottom,
            axis y line=left,
            ymode=log,
            minor tick style={draw=none},
            xlabel=\textbf{\small{Iterations}},
            ylabel=\textbf{\small{Sub-optimality}},
            xmax = 75,
            ymax = 0.5,
            ymin=1e-18,
            xmin = 0,
        ]
        \end{axis}
      }
      %
      %
      %
      \onslide<3-4> {
        \pgfplotscreateplotcyclelist{cycle_perfprofiles_synthetic_logistic}{
          TolLightGreen, very thick, smooth\\    
          TolDarkBrown, very thick, smooth\\
          TolLightRed, very thick, smooth\\
        }
        \begin{axis}[
            width   = 0.5\textwidth,
            height  = 0.4\textwidth,,
            legend style={
                at={(1.1,0.5)},
                anchor=west,
                legend columns=1,
                draw=none
            },
            cycle list name=cycle_perfprofiles_synthetic_logistic,
            mbaseplot,
            axis line style = ultra thick,
            major tick style = {ultra thick,color=mDarkTeal},
            xmajorgrids=true,
            ymajorgrids=true,
            major grid style={dotted},
            axis x line=bottom,
            axis y line=left,
            ymode=log,
            minor tick style={draw=none},
            xlabel=\textbf{\small{Iterations}},
            ylabel=\textbf{\small{Sub-optimality}},
            xmax = 75,
            ymax = 0.5,
            ymin=1e-18,
        ]
          \foreach \solver in {ProximalGradient,ProximalGradientScr}{
            \addplot table[
                x=grid_iter,
                y=\solver_iter_sopt,
                col sep=comma
            ] {data/quadratic_l1l2norm(0.5)_50_500_1000_0.5_10_True_0.3_600.0_1e-14.csv};
          }
        \end{axis}
      }
      %
      %
      %
      \onslide<5-> {
        \pgfplotscreateplotcyclelist{cycle_perfprofiles_synthetic_logistic}{
          TolLightGreen, very thick, smooth\\    
          TolDarkBrown, very thick, smooth\\
          TolLightRed, very thick, smooth\\
        }
        \begin{axis}[
            width   = 0.5\textwidth,
            height  = 0.4\textwidth,,
            legend style={
                at={(1.1,0.5)},
                anchor=west,
                legend columns=1,
                draw=none
            },
            cycle list name=cycle_perfprofiles_synthetic_logistic,
            mbaseplot,
            axis line style = ultra thick,
            major tick style = {ultra thick,color=mDarkTeal},
            xmajorgrids=true,
            ymajorgrids=true,
            major grid style={dotted},
            axis x line=bottom,
            axis y line=left,
            ymode=log,
            minor tick style={draw=none},
            xlabel=\textbf{\small{Iterations}},
            ylabel=\textbf{\small{Sub-optimality}},
            xmax = 75,
            ymax = 0.5,
            ymin=1e-18,
        ]

          \foreach \solver in {ProximalGradient,ProximalGradientScr,ProximalGradientScrRlx}{
            \addplot table[
                x=grid_iter,
                y=\solver_iter_sopt,
                col sep=comma
            ] {data/quadratic_l1l2norm(0.5)_50_500_1000_0.5_10_True_0.3_600.0_1e-14.csv};
            \label{plot:\solver}
          }
        \end{axis}
      } 
    \end{scope}
  \end{tikzpicture}
\end{frame}

\begin{frame}
  \begin{tikzpicture}[remember picture,overlay]
    \begin{scope}[xshift=0.5\linewidth]
      \onslide<1-> {
        \node at (-4.75,4) {\Large{\textbf{Let's recap}}};
        %
        %
        %
        \node[text width=0.45\linewidth, align=center] (problem) at (0,3.5) {
          \begin{blockcolor}{mDarkTeal}{Problem}
            \centering
            $\opt{\pobj} = \min_{\pv \in \kR^{\pdim}} \lfunc(\dic\pv) + \rfunc(\pv)$
          \end{blockcolor}
        };
        %
        \node[font=\small,anchor=west] (problem-text) at ($(problem)+(2.75,-0.15)$) {$\rfunc(\pv) = \reg\norm{\pv}{0} + \pfunc(\pv)$};
        \draw[ultra thick,->] ($(problem-text.west)+(-0.3,0)$) -- ($(problem-text.west)+(0.05,0)$);
        %
        %
        %
        \node (bnb) at ($(problem.south)+(0,-0.3)$) {};
        \draw[ultra thick,->] ($(bnb)+(0,0.4)$) -- ($(bnb)+(0,-0.4)$);
        \node[draw,ultra thick,font=\small,text width=0.25\linewidth,align=center,fill=mLightWhite,inner sep=2pt] at (bnb)  {BnB algorithm};
        %
        %
        %
        \node[text width=0.45\linewidth, align=center] (restrict) at ($(bnb)+(0,-0.9)$) {
          \begin{blockcolor}{mDarkTeal}{Restriction to region $\nodeSymb$}
            \centering
            $\node{\pobj} = \min_{\textcolor{mLightBrown}{\pv \in \nodeSymb}} \lfunc(\dic\pv) + \rfunc(\pv)$
          \end{blockcolor}
        };
        %
        \node[font=\small,anchor=west] (easy-text) at ($(restrict)+(2.75,-0.15)$) {lower bound on $\node{\pobj}$};
        \draw[ultra thick,->] ($(easy-text.west)+(-0.3,0)$) -- ($(easy-text.west)+(0.05,0)$);
        %
        %
        %
        \node (challenge) at ($(restrict.south)+(0,-0.3)$) {};
        \draw[ultra thick,->] ($(challenge)+(0,0.4)$) -- ($(challenge)+(0,-0.4)$);
        \node[draw,ultra thick,font=\small,text width=0.25\linewidth,align=center,fill=mLightWhite,inner sep=2pt] at (challenge) {standard strategy};
        %
        %
        %
        \node[text width=0.45\linewidth, align=center] (relax) at ($(challenge)+(0,-0.9)$) {
          \begin{blockcolor}{mDarkTeal}{Relaxation for region $\nodeSymb$}
            \centering
            $\node{\pobj}_{\text{lb}} = \min_{\pv \in \nodeSymb} \lfunc(\dic\pv) + \textcolor{mLightBrown}{\rfunc_{\text{lb}}}(\pv)$
          \end{blockcolor}
        };
        %
        \node[font=\small,anchor=west] (relax-text) at ($(relax)+(2.75,-0.15)$) {no generic construction};
        \draw[ultra thick,->] ($(relax-text.west)+(-0.3,0)$) -- ($(relax-text.west)+(0.05,0)$);
        %
        %
        %
        \begin{scope}[opacity=0.5]
          \node[mLightBrown] at (-4,-1.75) (axis1) {\textbf{Axis 1}};
          \node[text width=0.3\linewidth,align=center,mLightBrown,font=\small] at ($(axis1.south)+(0,-0.5)$) (axis1-text) {How to construct relaxations generically ?};
          \draw[ultra thick,mLightBrown] ($(axis1-text.center)+(-1.7,0.5)$) rectangle ($(axis1-text.center)+(1.7,-0.5)$);
          %
          \node[text width=0.35\linewidth,align=left,anchor=north,font=\small] at ($(axis1)+(0.1,-1.25)$) {1) Set $\rfunc_{\text{lb}}$ $=$ $\rfunc_{\text{cvx}}$ \\ 2) Closed-form expression \\ 3) Generalize BnB method};
        \end{scope}
      }
      %
      %
      %
      \onslide<1> {
        \node[mLightBrown] at (0,-1.75) (axis2) {\textbf{Axis 2}};
        \node[text width=0.3\linewidth,align=center,mLightBrown,font=\small] at ($(axis2.south)+(0,-0.5)$) (axis2-text) {How to solve relaxations efficiently ?};
        \draw[ultra thick,mLightBrown] ($(axis2-text.center)+(-1.7,0.5)$) rectangle ($(axis2-text.center)+(1.7,-0.5)$);
        %
        \node[text width=0.35\linewidth,align=left,anchor=north,font=\small] at ($(axis2)+(0.1,-1.25)$) {1) Cast as convex problem \\ 2) Screening/smoothing \\ 3) Reduce solving cost};
      }
      %
      %
      %
      \onslide<2-> {
        \begin{scope}[opacity=0.5]
          \node[mLightBrown] at (0,-1.75) (axis2) {\textbf{Axis 2}};
          \node[text width=0.3\linewidth,align=center,mLightBrown,font=\small] at ($(axis2.south)+(0,-0.5)$) (axis2-text) {How to solve relaxations efficiently ?};
          \draw[ultra thick,mLightBrown] ($(axis2-text.center)+(-1.7,0.5)$) rectangle ($(axis2-text.center)+(1.7,-0.5)$);
          %
          \node[text width=0.35\linewidth,align=left,anchor=north,font=\small] at ($(axis2)+(0.1,-1.25)$) {1) Cast as convex problem \\ 2) Screening/smoothing \\ 3) Reduce solving cost};
        \end{scope}
      }
      %
      %
      %
      \onslide<2-> {
        \node[mLightBrown] at (4,-1.75) (axis3) {\textbf{Axis 3}};
        \node[text width=0.3\linewidth,align=center,mLightBrown,font=\small] at ($(axis3.south)+(0,-0.5)$) (axis3-text) {How to improve the standard strategy ?};
        \draw[ultra thick,mLightBrown] ($(axis3-text.center)+(-1.7,0.5)$) rectangle ($(axis3-text.center)+(1.7,-0.5)$);
        %
        \node[text width=0.35\linewidth,align=center,anchor=north,font=\small] at ($(axis3)+(0,-1.25)$) {Manuscript -- Chap. 4-5};
        \node[text width=0.35\linewidth,align=left,anchor=north,font=\small] at ($(axis3)+(0.4,-1.75)$) {$\rightarrow$ ICASSP (2022) \\ \textcolor{mDarkTeal!25}{$\rightarrow$ EUSIPCO (2023)} \\ $\rightarrow$ ICML (2024)};
      }
    \end{scope}
  \end{tikzpicture}
\end{frame}