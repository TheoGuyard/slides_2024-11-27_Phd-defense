\section{Axis 2 -- How to solve relaxations efficiently ?}

\begin{frame}{Axis 2 -- Convex optimization}
  \begin{tikzpicture}[remember picture,overlay]
    \begin{scope}[xshift=0.5\linewidth]
      \onslide<1-> {
        \node[text width=0.4\linewidth, align=center] (relax) at (0,3.25) {
          \begin{blockcolor}{mDarkTeal}{Relaxation for region $\nodeSymb$}
            \centering
            $\min_{\pv \in \nodeSymb} \lfunc(\dic\pv) + \rfunc_{\text{cvx}}(\pv)$
          \end{blockcolor}
        };
      }
      %
      %
      %
      \onslide<2-> {
        \node[text width=0.4\linewidth, align=center] (problem) at ($(relax)+(0,-2.5)$) {
          \begin{blockcolor}{mDarkTeal}{Convex problem}
              \centering
              $\min_{\textcolor{TolLightOrange}{\pv \in \kR^{\pdim}}} \lfunc(\dic\pv) + \textcolor{TolLightOrange}{\relaxrfunc}(\pv)$
          \end{blockcolor}
        };
        %
        \draw[ultra thick,<->] ($(relax.south)+(0,0.1)$) -- ($(problem.north)+(0,-0.3)$) node[midway,draw,ultra thick,fill=TolLightWhite] {reformulation};
      }
      %
      %
      %
      \onslide<3-> {
          \node[text width=0.8\linewidth,align=left,anchor=west,font=\small] at ($(problem.east)+(0,-0.1)$) {$\relaxrfunc$ proper, closed, convex \\ $\relaxrfunc$ separable \\ $\relaxrfunc$ non-smooth at $\pv=\0$};
      }
      %
      %
      %
      \onslide<4-> {
          \node[text width=0.8\linewidth,align=left,anchor=west,font=\small] at ($(relax.east)+(0,-0.1)$) {remind $\rfunc_{\text{cvx}}$ acts as an \\  $\ell_1$-norm near $\pv = \0$};
      }
      %
      %
      %
      \onslide<5-> {
        \node[text width=0.4\linewidth,align=center] (beyond) at ($(problem)+(-4.25,0.8)$) {\textbf{Applications beyond \\ the BnB scope}};
        \draw[ultra thick,->] ($(beyond.south)+(0,0)$) .. controls ($(beyond.south)+(0,-0.4)$) .. ($(problem.west)+(0,-0.1)$);
      }
      %
      %
      %
      \onslide<6-> {
        \node[text width=0.4\linewidth,align=center,anchor=north] (method) at ($(problem)+(-3,-1.25)$) {\textbf{$\1^{\text{st}}$-order methods} \\ Proximal gradient \\ Coordinate descent \\ ...};
        %
        \node (x0) at ($(method.south)+(-2,-0.8)$) {$\iter{\pv}{0}$};
        \node[draw,ultra thick] (xk) at ($(method.south)+(0,-0.8)$) {$\iter{\pv}{k} \rightarrow \iter{\pv}{k+1}$};
        \node[yshift=6,font=\small] at (xk.north) {iteration};
        \node (xopt) at ($(method.south)+(2,-0.8)$) {};
        %
        \draw[ultra thick] (x0) -- (xk);
        \draw[ultra thick] (xk) -- ($(xk)!0.65!(xopt)$) -- ($(xk)!0.65!(xopt)+(0,-0.65)$) -- ($(x0)!0.35!(xk)+(0,-0.65)$) -- ($(x0)!0.35!(xk)$);
        \draw[ultra thick,->] ($(xk)!0.65!(xopt)+(0,-0.65)$) -- ($(xk)+(-0.2,-0.65)$);
        \node[right,yshift=1] at (xopt) {$\opt{\pv}$};
        \draw[ultra thick,->] (xk) -- (xopt);
      }
      %
      %
      %
      \onslide<7-> {
        \node[text width=0.5\linewidth,align=center,anchor=north] (cost) at ($(problem)+(3,-1.25)$) {\textbf{Solving cost}};
        \node[text width=0.5\linewidth,align=center] (cost-formula) at ($(cost)+(0,-1)$) {cost per iteration \\ $\times$ \\ number of iterations};
      }
      %
      %
      %
      \onslide<8-> {
        \node[TolLightOrange] at ($(cost-formula.south)+(-1.75,-0.75)$) {dimension of $\pv$};
        \draw[ultra thick,<-,TolLightOrange] ($(cost-formula.south)+(-1.75,-0.4)$) .. controls ($(cost-formula.south)+(-1.75,1.2)$) .. ($(cost-formula.south)+(-1.5,1.2)$);
      }
      %
      %
      %
      \onslide<9-> {
        \node[TolLightOrange] at ($(cost-formula.south)+(1.75,-0.75)$) {regularity of $\lfunc/\relaxrfunc$};
        \draw[ultra thick,<-,TolLightOrange] ($(cost-formula.south)+(1.75,-0.4)$) .. controls ($(cost-formula.south)+(1.75,0.2)$) .. ($(cost-formula.south)+(1.6,0.2)$);
      }
    \end{scope}
  \end{tikzpicture}
\end{frame}

\begin{frame}{Axis 2 -- Sparse structure exploitation}
  \begin{tikzpicture}[remember picture,overlay]
    \begin{scope}[xshift=0.5\linewidth]
      \onslide<1-> {
        \node[text width=0.35\linewidth, align=center] (problem) at (0,3.25) {
          \begin{blockcolor}{mDarkTeal}{Convex problem}
            \centering
            $\min_{\pv \in \kR^{\pdim}} \lfunc(\dic\pv) + \relaxrfunc(\pv)$
          \end{blockcolor}
        };
        %
        \node at ($(problem)+(0,-1.5)
        $) (question) {\textbf{Task}};
        \node[draw,ultra thick,text width=0.33\linewidth, align=center] at ($(question.south)+(0,-0.25)$) (question1-text) {Reduce the solving cost};
      }
      %
      %
      %
      \onslide<2-> {
        \node[text width=0.4\linewidth,align=center] (opt1) at ($(question)+(-3.5,-2)$) {\textbf{Option 1} \\ Reduce the dimension of $\pv$};
        \draw[->,ultra thick] ($(question1-text.west)+(0.05,0)$) .. controls ($(opt1)+(0,1)$) .. (opt1);
      }
      %
      %
      %
      \onslide<3-> {
        \node[font=\small] at ($(problem)+(4,0.3)
        $) (property) {$\relaxrfunc$ non-smooth at $\pv=\0$};
        \node[font=\small] at ($(property)+(0,-0.8)
        $) (property-text) {sparse solution $\opt{\pv}$};
        \draw[ultra thick,->] (property) -- (property-text);
      }
      %
      %
      %
      \onslide<4-> {
        \node[text width=\linewidth,align=center,anchor=north] (scr) at ($(opt1)+(0,-1)$) {\textbf{Screening tests} \\ Identify zeros in $\opt{\pv}$ \\ \small{L. El Ghaoui \textit{et al.} (2011)} \\ E. Ndiaye \textit{et al.} (2020)};
        \draw[ultra thick,->] (opt1) -- (scr);
      }
      %
      %
      %
      \onslide<5-> {
        \node[text width=0.4\linewidth,align=center] (opt2) at ($(question)+(3.5,-2)$) {\textbf{Option 2} \\ Improve the regularity of $\lfunc/\relaxrfunc$};
        \draw[->,ultra thick] ($(question1-text.east)+(-0.05,0)$) .. controls ($(opt2)+(0,1)$) .. (opt2);
      }
      %
      %
      %
      \onslide<6-> {
        \node[text width=\linewidth,align=center,anchor=north,TolLightOrange] (smt) at ($(opt2)+(0,-1)$) {\textbf{Smoothing tests} \\ Identify non-zeros in $\opt{\pv}$};
        \draw[ultra thick,->] (opt2) -- (smt);
      }
      %
      %
      %
      \onslide<7-> {
        \draw[<->,ultra thick,dashed] ($(scr.north)+(2,-0.25)$) -- ($(smt.north)+(-2,-0.25)$);
      }
    \end{scope}
  \end{tikzpicture}
\end{frame}

\begin{frame}{Axis 2 -- Screening and smoothing tests}
  \begin{tikzpicture}[remember picture,overlay]
    \begin{scope}[xshift=0.5\textwidth]
      \onslide<1-> {
        \node[text width=0.325\linewidth, align=center] (problem) at (-3.5,2.85) {
          \begin{blockcolor}{mDarkTeal}{Convex problem}
            \centering
            $\min_{\pv \in \kR^{\pdim}} \lfunc(\dic\pv) + \relaxrfunc(\pv)$
          \end{blockcolor}
        };
      }
      %
      %
      %
      \onslide<2-> {
        \node[align=center,text width=0.325\textwidth] (dual) at (-3.5,-0.7) {
          \begin{blockcolor}{mDarkTeal}{Dual problem}
              \centering
              $\max_{\dv \in \kR^{\ddim}} \dfunc(\dv)$
          \end{blockcolor}
        };
      }
      %
      %
      %
      \onslide<3-> {
        \node[align=center,text width=0.3\textwidth] (optcond) at (-3.5,0.935) {
          {\scriptsize{if strong duality holds}} \\
          $\transp{\dic}\opt{\dv} \in \subdiff\relaxrfunc(\opt{\pv})$
        };
        %
        \draw[ultra thick,->] ($(problem.south)+(0,0.05)$) -- (optcond.north) node[midway,fill=TolLightWhite,draw,ultra thick,font=\scriptsize,inner sep=2pt,yshift=2] {solution $\opt{\pv}$};
        %
        \draw[ultra thick,->] ($(dual.north)+(0,-0.3)$) -- (optcond.south) node[midway,fill=TolLightWhite,draw,ultra thick,font=\scriptsize,inner sep=2pt,yshift=-2] {solution $\opt{\dv}$};
      }
      %
      %
      %
      \onslide<4-> {
        \node (intermediate) at (2.75,3.5) {\textbf{Intermediate result}};
        \node[draw,ultra thick,text width=0.525\linewidth,align=center,anchor=north] at ($(intermediate.south)+(0,0)$) (result-text) {Some zeros and non-zeros in $\opt{\pv}$ can be identified from a dual solution $\opt{\dv}$.};
      }
      %
      %
      %
      \onslide<5> {
        \node[draw,ultra thick,dashed,fill=TolLightWhite,text width=0.3\linewidth,align=center] (safe) at (2.75,1.25) {\textbf{Safe region} \\ $\saferegion \subseteq \kR^{\ddim}$ with $\opt{\dv} \in \saferegion$};
      }
      %
      %
      %
      \onslide<6-> {
        \draw[ultra thick,->,dashed] ($(safe)+(0,0.85)$) -- ($(safe)+(0,-0.95)$);
        \node[draw,ultra thick,dashed,fill=TolLightWhite,text width=0.3\linewidth,align=center] (safe) at (2.75,1.25) {\textbf{Safe region} \\ $\saferegion \subseteq \kR^{\ddim}$ with $\opt{\dv} \in \saferegion$};
        %
        \node[TolLightOrange] (spotlight) at (2.75,0) {\textbf{Spotlight result}};
        \node[draw,ultra thick,text width=0.525\linewidth,align=center,anchor=north,TolLightOrange] at ($(spotlight.south)+(0,0)$) (result-text) {Some zeros and non-zeros in $\opt{\pv}$ can be identified from a safe region $\saferegion$.};
      }
      %
      %
      %
      \onslide<7-> {
        \node[font=\scriptsize] at ($(problem.north)+(0,-0.1)$) {remind $\relaxrfunc(\pv) = \sum_{\idxentry=1}^{\pdim}\separable{\relaxrfunc}{\idxentry}(\pvi{\idxentry})$};
      }
      \onslide<8-> {
        \node[text width=\linewidth,align=left] (theorem) at (0,-2.75) {
          \textbf{Theorem --} Given a safe region $\saferegion$, note $\transp{\dici{}}\saferegion = \kset{\transp{\dici{}}\dv}{\dv \in \saferegion}$, one has\vspace*{-5pt}
          \begin{alignat*}{6}
            \text{Screening test:} &&\quad \transp{\dici{\idxentry}}\saferegion &\subseteq \interior(\subdiff\separable{\relaxrfunc}{\idxentry}(0)) &&\implies \opt{\pvi{\idxentry}} = 0 \\
            \text{Smoothing test:} &&\quad \transp{\dici{\idxentry}}\saferegion &\subseteq \complset(\subdiff\separable{\relaxrfunc}{\idxentry}(0)) &&\implies \opt{\pvi{\idxentry}} \neq 0
          \end{alignat*}
        };
      }
      %
      %
      %
      \onslide<9-> {
        \draw[ultra thick,TolLightOrange] ($(theorem)+(-1.3,-0.8)$) -- ($(theorem)+(-1.3,-1)$) -- ($(theorem)+(2,-1)$) -- ($(theorem)+(2,-0.8)$);
        \node[TolLightOrange,font=\small] at (0.4,-4) (easy) {easy to evaluate if $\saferegion$ has a simple shape};
      }
    \end{scope}
  \end{tikzpicture}
\end{frame}

\begin{frame}{Axis 2 -- Dynamic identification}
  \begin{tikzpicture}[remember picture,overlay]
    \begin{scope}[xshift=0.5\linewidth]
      \onslide<1-> {
        \node[text width=0.325\linewidth, align=center] (problem) at (0,3.25) {
          \begin{blockcolor}{mDarkTeal}{Convex problem}
              \centering
              $\min_{\pv \in \kR^{\pdim}} \lfunc(\dic\pv) + \relaxrfunc(\pv)$
          \end{blockcolor}
        };
        %
        \node (x0) at (-2.75,1) {$\iter{\pv}{0}$};
        \node[draw,ultra thick] (xk) at (0,1) {$\iter{\pv}{k} \rightarrow \iter{\pv}{k+1}$};
        \node[yshift=6,font=\small] at (xk.north) {iteration};
        \node (xopt) at (2.75,1) {};
        \node[yshift=20,font=\small] at (xk.north) {\textbf{$\1^{st}$-order methods}};
        %
        \draw[ultra thick] (x0) -- (xk);
        \draw[ultra thick] (xk) -- ($(xk)!0.55!(xopt)$) -- ($(xk)!0.55!(xopt)+(0,-0.65)$) -- ($(x0)!0.45!(xk)+(0,-0.65)$) -- ($(x0)!0.45!(xk)$);
        \draw[ultra thick,->] ($(xk)!0.55!(xopt)+(0,-0.65)$) -- ($(xk)+(-0.2,-0.65)$);
        \node[right] at (xopt) {$\opt{\pv}$};
        \draw[ultra thick,->] (xk) -- (xopt);
      }
      %
      %
      %
      \onslide<2-> {
        \node[text width=0.5\linewidth,align=center,anchor=north,font=\small] (cost) at ($(problem)+(4,0.65)$) {\textbf{Solving cost}};
        \node[text width=0.5\linewidth,align=center,font=\small] (cost-formula) at ($(cost)+(0,-0.85)$) {cost per iteration \\ $\times$ \\ number of iterations};
      }
      %
      %
      %
      \onslide<3-> {
        \fill[TolLightWhite] ($(xk)!0.55!(xopt)+(0,-0.6)$) rectangle ($(x0)!0.45!(xk)+(0,-0.7)$);
        \fill[TolLightWhite] ($(xk)+(-0.5,-0.5)$) rectangle ($(xk)+(0.5,-0.8)$);
        %
        \draw[ultra thick] (xk) -- ($(xk)!0.55!(xopt)$) -- ($(xk)!0.55!(xopt)+(0,-2)$) -- ($(x0)!0.45!(xk)+(0,-2)$) -- ($(x0)!0.45!(xk)$);
        %
        \draw[ultra thick,->] ($(xk)!0.55!(xopt)$) -- ($(xk)!0.55!(xopt)+(0,-0.5)$);
        \draw[ultra thick,->] ($(xk)!0.55!(xopt)$) -- ($(xk)!0.55!(xopt)+(0,-1.75)$);
        %
        \draw[ultra thick,->] ($(x0)!0.45!(xk)+(0,-2)$) -- ($(x0)!0.45!(xk)+(0,-0.25)$);
        \draw[ultra thick,->] ($(x0)!0.45!(xk)+(0,-2)$) -- ($(x0)!0.45!(xk)+(0,-1.5)$);
        %
        \draw[ultra thick,->] ($(xk)!0.55!(xopt)+(0,-2)$) -- ($(xk)+(-0.2,-2)$);
      }
      %
      %
      %
      \onslide<4-> {
        \node[draw,ultra thick,fill=TolLightWhite,font=\small,text width=0.2\linewidth,align=center] (safe) at ($(xk)!0.55!(xopt)+(0,-1)$) {new safe region};
      }
      %
      %
      %
      \onslide<5-> {
        \node[draw,ultra thick,fill=TolLightWhite,font=\small,text width=0.2\linewidth,align=center] (scr) at ($(xk)+(0,-2)$) {screening test};
        %
        \node[font=\small] (identif-zero) at ($(scr)+(0,-0.75)$) {Identify some $\opt{\pvi{\idxentry}} = 0$};
      }
      %
      %
      %
      \onslide<6-> {
        \node[font=\small,anchor=north,text width=0.3\linewidth,align=center] (action-zero) at ($(identif-zero)+(0,-0.75)$) {Set $\pvi{\idxentry} = 0$ \\ \textcolor{TolLightOrange}{Reduce cost per iteration}};
        \draw[ultra thick,->] (identif-zero) -- (action-zero);
      }
      %
      %
      %
      \onslide<7-> {
        \node[draw,ultra thick,TolLightOrange,fill=TolLightWhite,font=\small,text width=0.2\linewidth,align=center] (smt) at ($(x0)!0.45!(xk)+(0,-1)$) {smoothing test};
        %
        \node[font=\small] (identif-nzero) at ($(smt)+(-2.9,0)$) {Identify some $\opt{\pvi{\idxentry}} \neq 0$};
      }
      %
      %
      %
      \onslide<8-> {
        \node[font=\small,text width=0.3\linewidth,align=center,anchor=north] (action-nzero) at ($(identif-nzero)+(0,-0.75)$) {Smooth $\separable{\relaxrfunc}{\idxentry}$, use $2^{\text{nd}}$-order iterations \\ \textcolor{TolLightOrange}{Reduce number \\ of iterations}};
        \draw[ultra thick,->] (identif-nzero) -- (action-nzero);
      }
      %
      %
      %
      \onslide<9-> {
        \node[font=\small] (enet) at ($(action-nzero)+(0,-1.25)$) {$\relaxrfunc(\pv) = \norm{\pv}{1} + \norm{\pv}{2}^2$};
        %
        \node[font=\small] (enet-smooth) at ($(enet)+(0,-1)$) {$\relaxrfunc(\pv) = \transp{\mathbf{s}}\pv + \norm{\pv}{2}^2$};
        %
        \draw[ultra thick,->] (enet) -- (enet-smooth);
      }
      %
      %
      %
      \onslide<10-> {
        \node[font=\small,text width=0.3\linewidth,align=center] (safe-action) at ($(safe)+(2.5,-2)$) {\textcolor{TolLightOrange}{All} zeros and non-zeros identified after a finite number of iterations};
        \draw[ultra thick,->,dashed] ($(safe-action)+(-1.25,2)$) .. controls ($(safe-action)+(0,2)$) .. (safe-action) node[midway,font=\small,xshift=12,yshift=12,rotate=-45,text width=0.2\linewidth,align=center] {additional \\ conditions};
      }
      %
      %
      %
      \onslide<11-> {
        \draw[dashed,ultra thick,->,text width=0.3\linewidth,align=center] ($(safe-action)+(0,-0.75)$) -- ($(safe-action)+(0,-1)$) node[below,font=\small] {Smooth problem of reduced dimension};
      }
    \end{scope}
  \end{tikzpicture}
\end{frame}

\begin{frame}{Axis 2 -- Numerics}
  \begin{tikzpicture}[remember picture,overlay]
      \begin{scope}[xshift=0.5\linewidth]
        \onslide<1-> {
          \node[text width=0.325\linewidth, align=center] (problem) at (0,3.25) {
            \begin{blockcolor}{mDarkTeal}{Convex problem}
                \centering
                $\min_{\pv \in \kR^{\pdim}} \lfunc(\dic\pv) + \relaxrfunc(\pv)$
            \end{blockcolor}
          };
          %
          \node[font=\small] at ($(problem)+(0,-1)$) {$\dic \in \kR^{100\times300}$ / $\lfunc$ : Logistic / $\relaxrfunc$ : Elastic-net};
        }
        %
        %
        %
        \onslide<4-6> {
          \node[text width=0.3\linewidth,align=center,font=\small] at (3,-1.5) {\textbf{$1^{\text{st}}$-order $\rightarrow$ $2^{\text{nd}}$-order} \\ Faster convergence};
        }
        %
        %
        %
        \onslide<5-6> {
          \node[text width=0.5\linewidth,align=center,font=\small] at (3,-2.25) {More expensive iterations};
        }
        %
        %
        %
        \onslide<6> {
          \node[text width=0.5\linewidth,align=center,font=\small,TolLightOrange] at (3,-2.75) {What about the solving time ?};
        }
      \end{scope}
      \begin{scope}[xshift=0.1\linewidth,yshift=0\textheight]
        \onslide<2-> {
          \node[font=\small,text width=0.5\linewidth,align=left] at (7.75,0.75) {\textbf{Accelerated proximal gradient} \\ \ref{plot:pg} Vanilla method \\ \ref{plot:pgs} With screening \\ \ref{plot:pgss} With screening and smoothing};
        }
        %
        %
        %
        \onslide<3-> {
          \pgfplotscreateplotcyclelist{cycle_perfprofiles_synthetic_logistic}{
              TolLightGreen, very thick, smooth\\    
              TolLightBrown, very thick, smooth\\
              TolLightRed, very thick, smooth\\
          }
          \begin{axis}[
              width   = 0.5\textwidth,
              height  = 0.35\textwidth,,
              legend style={
                  at={(1.1,0.5)},
                  anchor=west,
                  legend columns=1,
                  draw=none
              },
              cycle list name=cycle_perfprofiles_synthetic_logistic,
              mbaseplot,
              axis line style = ultra thick,
              major tick style = {ultra thick,color=mDarkTeal},
              xmajorgrids=true,
              ymajorgrids=true,
              major grid style={dotted},
              axis x line=bottom,
              axis y line=left,
              ymode=log,
              minor tick style={draw=none},
              xlabel=\textbf{\small{Iterations}},
              ylabel=\textbf{\small{Sub-optimality}},
              xmax = 175,
              ymax = 5e3,
              ymin=1e-10,
          ]

              \foreach \solver in {pg,pgs,pgss}{
                  \addplot table[
                      x=iter,
                      y=\solver_iter_gval,
                      col sep=comma
                  ] {data/pg_logistic_enet_5_100_300_0.25_0.1.csv};
                  \label{plot:\solver}
              }
          \end{axis}
        }
      \end{scope}
      %
      %
      %
      \onslide<5-> {
        \begin{scope}[xshift=0.1\linewidth,yshift=-0.35\textheight]
            \pgfplotscreateplotcyclelist{cycle_perfprofiles_synthetic_logistic}{
                TolLightGreen, very thick, smooth\\    
                TolLightBrown, very thick, smooth\\
                TolLightRed, very thick, smooth\\
            }
            \begin{axis}[
                width   = 0.5\textwidth,
                height  = 0.35\textwidth,,
                legend style={
                    at={(1.1,0.5)},
                    anchor=west,
                    legend columns=1,
                    draw=none,
                },
                cycle list name=cycle_perfprofiles_synthetic_logistic,
                mbaseplot,
                axis line style = ultra thick,
                major tick style = {ultra thick,color=mDarkTeal},
                xmajorgrids=true,
                ymajorgrids=true,
                major grid style={dotted},
                axis x line=bottom,
                axis y line=left,
                ymode=log,
                minor tick style={draw=none},
                xlabel=\textbf{\small{Iterations}},
                ylabel=\textbf{\small{Iteration cost}},
                y label style={yshift=11pt},
                xmax = 175,
                ymax = 2e5,
                ymin = 3e3,
            ]

                \foreach \solver in {pg,pgs,pgss}{
                    \addplot table[
                        x=iter,
                        y=\solver_iter_cost,
                        col sep=comma
                    ] {data/pg_logistic_enet_5_100_300_0.25_0.1.csv};
                }
            \end{axis}
        \end{scope}
      }
      %
      %
      %
      \onslide<7-> {
        \begin{scope}[xshift=0.65\linewidth,yshift=-0.35\textheight]
            \pgfplotscreateplotcyclelist{cycle_perfprofiles_synthetic_logistic}{
                TolLightGreen, very thick, smooth\\    
                TolLightBrown, very thick, smooth\\
                TolLightRed, very thick, smooth\\
            }
            \begin{axis}[
                width   = 0.5\textwidth,
                height  = 0.35\textwidth,,
                legend style={
                    at={(1.1,0.5)},
                    anchor=west,
                    legend columns=1,
                    draw=none
                },
                cycle list name=cycle_perfprofiles_synthetic_logistic,
                mbaseplot,
                axis line style = ultra thick,
                major tick style = {ultra thick,color=mDarkTeal},
                xmajorgrids=true,
                ymajorgrids=true,
                major grid style={dotted},
                axis x line=bottom,
                axis y line=left,
                ymode=log,
                xmode=log,
                minor tick style={draw=none},
                xlabel=\textbf{\small{Time (seconds)}},
                ylabel=\textbf{\small{Sub-optimality}},
                ymax = 5e3,
                ymin=1e-10,
                xmax = 0.2,
            ]

                \foreach \solver in {pg,pgs,pgss}{
                    \addplot table[
                        x=time,
                        y=\solver_time_gval,
                        col sep=comma
                    ] {data/pg_logistic_enet_5_100_300_0.25_0.1.csv};
                }
            \end{axis}
        \end{scope}
      }
  \end{tikzpicture}
\end{frame}


\begin{frame}
  \begin{tikzpicture}[remember picture,overlay]
    \begin{scope}[xshift=0.5\linewidth]
      \onslide<1-> {
        \node at (-4.75,4) {\Large{\textbf{Let's recap}}};
        %
        \node[text width=0.45\linewidth, align=center] (problem) at (0,3.75) {
          \begin{blockcolor}{mDarkTeal}{Problem}
            \centering
            $\opt{\pobj} = \min_{\pv \in \kR^{\pdim}} \lfunc(\dic\pv) + \rfunc(\pv)$
          \end{blockcolor}
        };
        %
        \node[font=\small,anchor=west] (problem-text2) at ($(problem)+(2.75,0.15)$) {$\rfunc(\pv) = \reg\norm{\pv}{0} + \pfunc(\pv)$};
        \draw[ultra thick,->] ($(problem-text2.west)+(-0.3,0)$) -- ($(problem-text2.west)+(0.05,0)$);
        %
        \node[font=\small,anchor=west] (problem-text1) at ($(problem)+(2.75,-0.4)$) {many applications};
        \draw[ultra thick,->] ($(problem-text1.west)+(-0.3,0)$) -- ($(problem-text1.west)+(0.05,0)$);
        %
        \node (bnb) at ($(problem.south)+(0,-0.3)$) {};
        \draw[ultra thick,->] ($(bnb)+(0,0.4)$) -- ($(bnb)+(0,-0.4)$);
        \node[draw,ultra thick,font=\small,text width=0.25\linewidth,align=center,fill=TolLightWhite,inner sep=2pt] at (bnb)  {BnB algorithm};
        %
        \node[text width=0.45\linewidth, align=center] (restrict) at ($(bnb)+(0,-0.9)$) {
          \begin{blockcolor}{mDarkTeal}{Restriction to region $\nodeSymb$}
            \centering
            $\node{\pobj} = \min_{\textcolor{TolLightOrange}{\pv \in \nodeSymb}} \lfunc(\dic\pv) + \rfunc(\pv)$
          \end{blockcolor}
        };
        %
        \node[font=\small,anchor=west] (easy-text1) at ($(restrict)+(2.75,0.15)$) {pruning test};
        \draw[ultra thick,->] ($(easy-text1.west)+(-0.3,0)$) -- ($(easy-text1.west)+(0.05,0)$);
        %
        \node[font=\small,anchor=west] (easy-text2) at ($(restrict)+(2.75,-0.4)$) {lower bound on $\node{\pobj}$};
        \draw[ultra thick,->] ($(easy-text2.west)+(-0.3,0)$) -- ($(easy-text2.west)+(0.05,0)$);
        %
        \node (challenge) at ($(restrict.south)+(0,-0.3)$) {};
        \draw[ultra thick,->] ($(challenge)+(0,0.4)$) -- ($(challenge)+(0,-0.4)$);
        \node[draw,ultra thick,font=\small,text width=0.25\linewidth,align=center,fill=TolLightWhite,inner sep=2pt] at (challenge) {standard strategy};
        %
        %
        %
        \node[text width=0.45\linewidth, align=center] (relax) at ($(challenge)+(0,-0.9)$) {
          \begin{blockcolor}{mDarkTeal}{Relaxation for region $\nodeSymb$}
            \centering
            $\node{\pobj}_{\text{lb}} = \min_{\pv \in \nodeSymb} \lfunc(\dic\pv) + \textcolor{TolLightOrange}{\rfunc_{\text{lb}}}(\pv)$
          \end{blockcolor}
        };
        %
        \node[font=\small,anchor=west,text width=0.3\linewidth,align=left] (relax-text) at ($(relax)+(2.75,-0.1)$) {instance-specific \\ construction of $\rfunc_{\text{lb}}$};
        \draw[ultra thick,->] ($(relax-text.west)+(-0.3,0)$) -- ($(relax-text.west)+(0.05,0)$);
        %
        %
        %
        \begin{scope}[opacity=0.5]
          \node[TolLightOrange] at (-4,-1.5) (axis1) {\textbf{Axis 1}};
          \node[text width=0.3\linewidth,align=center,TolLightOrange,font=\small] at ($(axis1.south)+(0,-0.5)$) (axis1-text) {How to construct relaxations generically ?};
          \draw[ultra thick,TolLightOrange] ($(axis1-text.center)+(-1.7,0.5)$) rectangle ($(axis1-text.center)+(1.7,-0.5)$);
          %
          \node[text width=0.35\linewidth,align=left,anchor=north,font=\small] at ($(axis1)+(0.1,-1.25)$) {1) Set $\rfunc_{\text{lb}}$ $=$ $\rfunc_{\text{cvx}}$ \\ 2) Closed-form expression \\ 3) Generalize BnB method};
        \end{scope}
      }
      %
      %
      %
      \onslide<1> {
        \node[TolLightOrange] at (0,-1.5) (axis2) {\textbf{Axis 2}};
        \node[text width=0.3\linewidth,align=center,TolLightOrange,font=\small] at ($(axis2.south)+(0,-0.5)$) (axis2-text) {How to solve relaxations efficiently ?};
        \draw[ultra thick,TolLightOrange] ($(axis2-text.center)+(-1.7,0.5)$) rectangle ($(axis2-text.center)+(1.7,-0.5)$);
        %
        \node[text width=0.35\linewidth,align=left,anchor=north,font=\small] at ($(axis2)+(0.1,-1.25)$) {1) Cast as convex problem \\ 2) Screening/smoothing \\ 3) Reduce solving cost};
      }
      %
      %
      %
      \onslide<2-> {
        \begin{scope}[opacity=0.5]
          \node[TolLightOrange] at (0,-1.5) (axis2) {\textbf{Axis 2}};
          \node[text width=0.3\linewidth,align=center,TolLightOrange,font=\small] at ($(axis2.south)+(0,-0.5)$) (axis2-text) {How to solve relaxations efficiently ?};
          \draw[ultra thick,TolLightOrange] ($(axis2-text.center)+(-1.7,0.5)$) rectangle ($(axis2-text.center)+(1.7,-0.5)$);
          %
          \node[text width=0.35\linewidth,align=left,anchor=north,font=\small] at ($(axis2)+(0.1,-1.25)$) {1) Cast as convex problem \\ 2) Screening/smoothing \\ 3) Reduce solving cost};
        \end{scope}
      }
      %
      %
      %
      \onslide<2-> {
        \node[TolLightOrange] at (4,-1.5) (axis3) {\textbf{Axis 3}};
        \node[text width=0.3\linewidth,align=center,TolLightOrange,font=\small] at ($(axis3.south)+(0,-0.5)$) (axis3-text) {How to improve the standard strategy ?};
        \draw[ultra thick,TolLightOrange] ($(axis3-text.center)+(-1.7,0.5)$) rectangle ($(axis3-text.center)+(1.7,-0.5)$);
        %
        \node[text width=0.35\linewidth,align=center,anchor=north,font=\small] at ($(axis3)+(0,-1.25)$) {Manuscript -- Chap. 4-5};
        \node[text width=0.35\linewidth,align=left,anchor=north,font=\small] at ($(axis3)+(0.4,-1.75)$) {$\rightarrow$ ICASSP (2022) \\ \textcolor{mDarkTeal!25}{$\rightarrow$ EUSIPCO (2023)} \\ $\rightarrow$ ICML (2024)};
      }
    \end{scope}
  \end{tikzpicture}
\end{frame}