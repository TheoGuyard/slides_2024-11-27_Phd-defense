\section{Axis 1 -- How to construct relaxations generically ?}

\begin{frame}{Axis 1 -- Generic relaxation construction}
  \begin{tikzpicture}[remember picture,overlay]
    \begin{scope}[xshift=0.5\linewidth]
      \onslide<1-> {
        \node[text width=0.45\linewidth, align=center] (restrict) at (0,3) {
          \begin{blockcolor}{mDarkTeal}{Restriction to region $\nodeSymb$}
              \centering
              $\node{\pobj} = \min_{\pv \in \nodeSymb} \lfunc(\dic\pv) + \rfunc(\pv)$
          \end{blockcolor}
        };
        %
        \node[font=\small,text width=0.4\linewidth,align=left,anchor=west] at ($(restrict)+(2.5,-0.1)$) {$\rfunc(\pv) = \reg\norm{\pv}{0} + \pfunc(\pv)$ \\ lower bound on $\node{\pobj}$};
      }
      %
      %
      %
      \onslide<2-> {
        \node[text width=0.45\linewidth, align=center] (relax) at (0,0.75) {
          \begin{blockcolor}{mDarkTeal}{Relaxation for region $\nodeSymb$}
              \centering
              $\node{\pobj}_{\text{lb}} = \min_{\pv \in \nodeSymb} \lfunc(\dic\pv) + \textcolor{TolLightOrange}{\rfunc_{\text{lb}}}(\pv)$
          \end{blockcolor}
        };
        %
        \node[font=\small,text width=0.4\linewidth,align=left,anchor=west] at ($(relax)+(2.5,-0.1)$) {$\rfunc_{\text{lb}} \leq \rfunc$ \\ $\rfunc_{\text{lb}}$ convex};
        %
        \draw[ultra thick,->] ($(restrict)+(0,-0.75)$) -- ($(relax)+(0,0.5)$) node[midway,fill=TolLightWhite,draw,ultra thick] {standard strategy};
        %
      }
      %
      %
      %
      \onslide<3-> {
        \node[text width=0.6\linewidth,align=center,font=\small] (l2norm) at ($(relax)+(-4,-2.5)$) {$\pfunc$ : $\ell_2$-norm \\ {M. Pilanci \textit{et al.} (2016)}};
        \draw[ultra thick,->] ($(relax)+(-1,-1)$) .. controls ($(relax)+(-3.5,-1.5)$) .. (l2norm);
        %
        \node[text width=0.6\linewidth,align=center,font=\small] (bound) at ($(relax)+(0,-2.5)$) {$\pfunc$ : bound cstr. \\ {S. Bourguignon \textit{et al.} (2020)}};
        \draw[ultra thick,->] ($(relax)+(0,-1)$) .. controls ($(relax)+(0,-2)$) .. (bound);
        %
        \node[text width=0.6\linewidth,align=center,font=\small] (l2normbound) at ($(relax)+(4,-2.5)$) {$\pfunc$ : $\ell_2$-norm + bound cstr. \\ {H. Hazimeh \textit{et al.} (2021)}};
        \draw[ultra thick,->] ($(relax)+(1,-1)$) .. controls ($(relax)+(3.5,-1.5)$) .. (l2normbound);
      }
      %
      %
      %
      \onslide<4-> {
        \draw[ultra thick] (-6,-2) -- (-6,-2.25) -- (6,-2.25) -- (6,-2);
        \node at (0,-2.75) {\textbf{Generalization by setting \textcolor{TolLightOrange}{$\boldsymbol{\rfunc}_{\text{lb}} = \boldsymbol{\rfunc}_{\text{cvx}}$}}};
      }
      %
      %
      %
      \onslide<5-> {
        \node[text width=0.75\linewidth,align=left] at (0.75,-3.6) {$\rightarrow$ $\pfunc$ proper, closed, convex \\ $\rightarrow$ $\pfunc$ separable, even, supercoercive, $\pfunc \geq \pfunc(\0) = 0$};
      }
    \end{scope}
  \end{tikzpicture}
\end{frame}

\begin{frame}{Axis 1 -- Convex envelope characterization}
  \begin{tikzpicture}[remember picture,overlay]
    \begin{scope}[xshift=0.5\linewidth]
      \onslide<1-> {
        \node[TolLightOrange] at (0,3.5) (result) {\textbf{Spotlight result}};
        \node[draw,ultra thick,TolLightOrange] at ($(result.south)+(0,-0.25)$) (result-text) {The convex envelope of $\rfunc(\pv) = \reg\norm{\pv}{0} + \pfunc(\pv)$ admits a closed-form expression.};
      }
      %
      %
      %
      \onslide<2-> {
        \node[text width=\linewidth,align=left] (theorem) at (-0.3,0.75) {\textbf{Theorem (1d version) --} Let $\rfunc(\pvi{}) = \reg\norm{\pvi{}}{0} + \pfunc(\pvi{})$, one has
        \begin{equation*}
          \textcolor{TolLightOrange}{\rfunc_{\text{cvx}}}(\pvi{}) = 
          \begin{cases}
            \rslope\abs{\pvi{}} &\text{if } \abs{\pvi{}} \leq \rlimit \\
            \pfunc(\pvi{}) + \reg &\text{otherwise}
          \end{cases}
        \end{equation*}
        where $(\rslope,\rlimit)$ are some ``easy-to-compute'' quantities.};
      }
      %
      %
      %
      \onslide<3-> {
        \begin{scope}[xshift=0.375\linewidth,yshift=-0.05\textheight,scale=0.9]
          \onslide<3-> {
            \node[right,LavenderBlush4] at (-1.5,1.6) {$\rfunc$};
            \draw[ultra thick,->] (-1.5,0) -- (1.5,0) node[right] {$\pvi{}$};
            \draw[ultra thick,->] (0,-0.2) -- (0,1.5);
            \node[above left] at (0,0.5) {$\reg$};
            %
            \draw[domain=-1.75:-0.1,smooth,variable=\x,LavenderBlush4,ultra thick] plot ({\x}, {0.5 + 0.5*\x*\x});
            \draw[domain=0.1:1.75,smooth,variable=\x,LavenderBlush4,ultra thick] plot ({\x}, {0.5 + 0.5*\x*\x});
            \fill[LavenderBlush4] (0,0) circle (0.1);
            \draw[LavenderBlush4,ultra thick] (0,0.5) circle (0.1);
          } 
          %
          %
          %
          \onslide<4-> {
            \draw[very thick,dashed] (0,0) -- (1,0.95);
            \draw[very thick,dashed] (1,1) -- (1,0);
            \fill[LavenderBlush4] (0,0) circle (0.1);
            \fill[mDarkTeal] (1,0.95) circle (0.1);
          }
          %
          %
          %
          \onslide<5-> {
            \draw[very thick,dashed] (0.3,0.3) .. controls (0.45,0.2) .. (0.5,0) node[midway,right] {$\rslope$};
            \node at (1,-0.25) {$\rlimit$};
          }
          %
          %
          %
          \onslide<6-> {
            \draw[ultra thick,TolLightOrange,dashed] (-1,0.95) -- (0,0);
            \draw[ultra thick,TolLightOrange,dashed] (0,0) -- (1,0.95);
            \draw[domain=1:1.75,smooth,variable=\x,TolLightOrange,dashed,ultra thick] plot ({\x}, {0.45 + 0.5*\x*\x});
            \draw[domain=-1.75:-1,smooth,variable=\x,TolLightOrange,dashed,ultra thick] plot ({\x}, {0.45 + 0.5*\x*\x});
            \fill[LavenderBlush4] (0,0) circle (0.1);
            \fill[mDarkTeal] (1,0.95) circle (0.1);
            \node[right,TolLightOrange] at (-1.75,0.75) {$\rfunc_{\text{cvx}}$};
          }
        \end{scope}
      }
      %
      %
      %
      \onslide<7-> {
        \node (generic) at (-3,-2) {\textbf{Generic relaxation construction}};
        \node [text width=0.5\linewidth, align=center,font=\small] at ($(generic)+(0,-0.75)$) {Characterize $\rfunc_{\text{lb}}$ $=$ $\rfunc_{\text{cvx}}$ \\ Encompasses prior contributions};
        \draw[ultra thick,->] ($(generic.north)+(0,0.5)$) -- ($(generic.north)+(0,0.1)$);
      }
      %
      %
      %
      \onslide<8-> {
        \node (practical) at (3,-2) {\textbf{Practical relaxation construction}};
        \node [text width=0.5\linewidth, align=center,font=\small] at ($(practical)+(0,-0.75)$) {Closed-form for $\subdiff\rfunc_{\text{cvx}}$ and $\prox_{\rfunc_{\text{cvx}}}$ \\ Enables standard solution methods};
        \draw[ultra thick,->] ($(practical.north)+(0,0.5)$) -- ($(practical.north)+(0,0.1)$);
      }
    \end{scope}
  \end{tikzpicture}
\end{frame}

\begin{frame}{Axis 1 -- Numerics}
  \begin{tikzpicture}[remember picture,overlay]
    \begin{scope}[xshift=0.5\linewidth]
      \onslide<1-> {
        \node[text width=0.45\linewidth, align=center] (problem) at (0,2.75) {
          \begin{blockcolor}{mDarkTeal}{Problem}
              \centering
              $\min_{\pv \in \kR^{\pdim}} \lfunc(\dic\pv) + \reg\norm{\pv}{0} + \pfunc(\pv)$
          \end{blockcolor}
        };
        %
        \node[font=\small] at ($(problem)+(0,-1)$) {$\dic \in \kR^{100\times300}$ / $\lfunc$ : Logistic / $\pfunc$ : Bound cstr. + $\ell_1$-norm};
        %
        \node[text width=0.6\linewidth,align=left,font=\small] at (3,-0.75) {\textbf{MIP solvers} \\ \ref{plot:bnb-Scip} \textcolor{mDarkTeal!25}{Scip} \\ \ref{plot:bnb-Cplex} \textcolor{mDarkTeal!25}{Cplex} \\ \ref{plot:bnb-Mosek} Mosek \\ \textbf{BnB solvers} \\ \ref{plot:bnb-L0Bnb} \textcolor{mDarkTeal!25}{L0Bnb} \\ \ref{plot:bnb-SBnb} \textcolor{mDarkTeal!25}{SBnb}};
      }
      %
      %
      %
      \onslide<2-> {
        \node[text width=0.6\linewidth,align=left,font=\small] at (3,-2.55) {\ref{plot:bnb-El0ps} El0ps};
      }
      %
      %
      %
      \onslide<3-> {
        \node[text width=0.3\linewidth,align=center] (obs1) at (4,-1) {El0ps is a \textcolor{TolLightOrange}{generic} BnB solver with \textcolor{TolLightOrange}{state-of-the-art} performance};
      }
    \end{scope}
    %
    %
    %
    \onslide<1> {
      \begin{scope}[xshift=0.0725\linewidth,yshift=-0.25\textheight]
        \pgfplotscreateplotcyclelist{cycle_perfprofiles_synthetic_logistic}{
          TolLightGreen, very thick, smooth\\
          TolDarkRed, very thick, smooth\\
        }
        \begin{axis}[
            width   = 0.5\textwidth,
            height  = 0.4\textwidth,,
            legend style={
                at={(1.1,0.5)},
                anchor=west,
                legend columns=1,
                draw=none
            },
            cycle list name=cycle_perfprofiles_synthetic_logistic,
            mbaseplot,
            axis line style = ultra thick,
            major tick style = {ultra thick,color=mDarkTeal},
            xmajorgrids=true,
            ymajorgrids=true,
            major grid style={dotted},
            axis x line=bottom,
            axis y line=left,
            xmode=log,
            minor tick style={draw=none},
            xlabel=\textbf{\small{Time (seconds)}},
            ymax=110,
            xmax=7000,
            ylabel=\textbf{\small{Instances solved}},
            ytick={0,25,50,75,100},
        ]

          \foreach \solver in {Mosek}{
            \addplot table[
                x=times,
                y expr=100*\thisrow{\solver},
                col sep=comma
            ] {data/perfprofiles_medium_Logistic_L1norm_3600.csv};
          }
        \end{axis}
        %
        \draw[very thick,dotted] (3.6,0) -- (3.6,2.65) node[above,font=\scriptsize] {1 hour};
      \end{scope}
    }
    %
    %
    %
    \onslide<2-> {
      \begin{scope}[xshift=0.0725\linewidth,yshift=-0.25\textheight]
        \pgfplotscreateplotcyclelist{cycle_perfprofiles_synthetic_logistic}{
          TolLightGreen, very thick, smooth\\
          TolDarkRed, very thick, smooth\\
        }
        \begin{axis}[
            width   = 0.5\textwidth,
            height  = 0.4\textwidth,,
            legend style={
                at={(1.1,0.5)},
                anchor=west,
                legend columns=1,
                draw=none
            },
            cycle list name=cycle_perfprofiles_synthetic_logistic,
            mbaseplot,
            axis line style = ultra thick,
            major tick style = {ultra thick,color=mDarkTeal},
            xmajorgrids=true,
            ymajorgrids=true,
            major grid style={dotted},
            axis x line=bottom,
            axis y line=left,
            xmode=log,
            minor tick style={draw=none},
            xlabel=\textbf{\small{Time (seconds)}},
            ymax=110,
            xmax=7000,
            ylabel=\textbf{\small{Instances solved}},
            ytick={0,25,50,75,100},
        ]

          \foreach \solver in {Mosek,El0ps}{
            \addplot table[
                x=times,
                y expr=100*\thisrow{\solver},
                col sep=comma
            ] {data/perfprofiles_medium_Logistic_L1norm_3600.csv};
            \label{plot:bnb-\solver}
          }
        \end{axis}
        %
        \draw[very thick,dotted] (3.6,0) -- (3.6,2.65) node[above,font=\scriptsize] {1 hour};
      \end{scope}
    }
  \end{tikzpicture}
\end{frame}

\begin{frame}
  \begin{tikzpicture}[remember picture,overlay]
    \begin{scope}[xshift=0.5\linewidth]
      \onslide<1-> {
        \node at (-4.75,4) {\Large{\textbf{Let's recap}}};
        %
        \node[text width=0.45\linewidth, align=center] (problem) at (0,3.75) {
          \begin{blockcolor}{mDarkTeal}{Problem}
            \centering
            $\opt{\pobj} = \min_{\pv \in \kR^{\pdim}} \lfunc(\dic\pv) + \rfunc(\pv)$
          \end{blockcolor}
        };
        %
        \node[font=\small,anchor=west] (problem-text2) at ($(problem)+(2.75,0.15)$) {$\rfunc(\pv) = \reg\norm{\pv}{0} + \pfunc(\pv)$};
        \draw[ultra thick,->] ($(problem-text2.west)+(-0.3,0)$) -- ($(problem-text2.west)+(0.05,0)$);
        %
        \node[font=\small,anchor=west] (problem-text1) at ($(problem)+(2.75,-0.4)$) {many applications};
        \draw[ultra thick,->] ($(problem-text1.west)+(-0.3,0)$) -- ($(problem-text1.west)+(0.05,0)$);
        %
        \node (bnb) at ($(problem.south)+(0,-0.3)$) {};
        \draw[ultra thick,->] ($(bnb)+(0,0.4)$) -- ($(bnb)+(0,-0.4)$);
        \node[draw,ultra thick,font=\small,text width=0.25\linewidth,align=center,fill=TolLightWhite,inner sep=2pt] at (bnb)  {BnB algorithm};
        %
        \node[text width=0.45\linewidth, align=center] (restrict) at ($(bnb)+(0,-0.9)$) {
          \begin{blockcolor}{mDarkTeal}{Restriction to region $\nodeSymb$}
            \centering
            $\node{\pobj} = \min_{\textcolor{TolLightOrange}{\pv \in \nodeSymb}} \lfunc(\dic\pv) + \rfunc(\pv)$
          \end{blockcolor}
        };
        %
        \node[font=\small,anchor=west] (easy-text1) at ($(restrict)+(2.75,0.15)$) {pruning test};
        \draw[ultra thick,->] ($(easy-text1.west)+(-0.3,0)$) -- ($(easy-text1.west)+(0.05,0)$);
        %
        \node[font=\small,anchor=west] (easy-text2) at ($(restrict)+(2.75,-0.4)$) {lower bound on $\node{\pobj}$};
        \draw[ultra thick,->] ($(easy-text2.west)+(-0.3,0)$) -- ($(easy-text2.west)+(0.05,0)$);
        %
        \node (challenge) at ($(restrict.south)+(0,-0.3)$) {};
        \draw[ultra thick,->] ($(challenge)+(0,0.4)$) -- ($(challenge)+(0,-0.4)$);
        \node[draw,ultra thick,font=\small,text width=0.25\linewidth,align=center,fill=TolLightWhite,inner sep=2pt] at (challenge) {standard strategy};
        %
        %
        %
        \node[text width=0.45\linewidth, align=center] (relax) at ($(challenge)+(0,-0.9)$) {
          \begin{blockcolor}{mDarkTeal}{Relaxation for region $\nodeSymb$}
            \centering
            $\node{\pobj}_{\text{lb}} = \min_{\pv \in \nodeSymb} \lfunc(\dic\pv) + \textcolor{TolLightOrange}{\rfunc_{\text{lb}}}(\pv)$
          \end{blockcolor}
        };
        %
        \node[font=\small,anchor=west,text width=0.3\linewidth,align=left] (relax-text) at ($(relax)+(2.75,-0.1)$) {instance-specific \\ construction of $\rfunc_{\text{lb}}$};
        \draw[ultra thick,->] ($(relax-text.west)+(-0.3,0)$) -- ($(relax-text.west)+(0.05,0)$);
      }
      %
      %
      %
      \onslide<1> {
        \node[TolLightOrange] at (-4,-1.5) (axis1) {\textbf{Axis 1}};
        \node[text width=0.3\linewidth,align=center,TolLightOrange,font=\small] at ($(axis1.south)+(0,-0.5)$) (axis1-text) {How to construct relaxations generically ?};
        \draw[ultra thick,TolLightOrange] ($(axis1-text.center)+(-1.7,0.5)$) rectangle ($(axis1-text.center)+(1.7,-0.5)$);
        %
        \node[text width=0.35\linewidth,align=left,anchor=north,font=\small] at ($(axis1)+(0.1,-1.25)$) {1) Set $\rfunc_{\text{lb}}$ $=$ $\rfunc_{\text{cvx}}$ \\ 2) Closed-form expression \\ 3) Generalize BnB method};
      }
      %
      %
      %
      \onslide<2-> {
        \begin{scope}[opacity=0.5]
          \node[TolLightOrange] at (-4,-1.5) (axis1) {\textbf{Axis 1}};
          \node[text width=0.3\linewidth,align=center,TolLightOrange,font=\small] at ($(axis1.south)+(0,-0.5)$) (axis1-text) {How to construct relaxations generically ?};
          \draw[ultra thick,TolLightOrange] ($(axis1-text.center)+(-1.7,0.5)$) rectangle ($(axis1-text.center)+(1.7,-0.5)$);
          %
          \node[text width=0.35\linewidth,align=left,anchor=north,font=\small] at ($(axis1)+(0.1,-1.25)$) {1) Set $\rfunc_{\text{lb}}$ $=$ $\rfunc_{\text{cvx}}$ \\ 2) Closed-form expression \\ 3) Generalize BnB method};
        \end{scope}
      }
      %
      %
      %
      \onslide<2-> {
        \node[TolLightOrange] at (0,-1.5) (axis2) {\textbf{Axis 2}};
        \node[text width=0.3\linewidth,align=center,TolLightOrange,font=\small] at ($(axis2.south)+(0,-0.5)$) (axis2-text) {How to solve relaxations efficiently ?};
        \draw[ultra thick,TolLightOrange] ($(axis2-text.center)+(-1.7,0.5)$) rectangle ($(axis2-text.center)+(1.7,-0.5)$);
        %
        \node[text width=0.35\linewidth,align=center,anchor=north,font=\small] at ($(axis2)+(0,-1.25)$) {Manuscript -- Chap. 6};
        \node[text width=0.35\linewidth,align=left,anchor=north,font=\small] at ($(axis2)+(0.20,-1.75)$) {$\rightarrow$ ICASSP (2022) \\ $\rightarrow$ Journal paper (202x)};
      }
    \end{scope}
  \end{tikzpicture}
\end{frame}
