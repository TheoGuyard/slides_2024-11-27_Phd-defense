\documentclass[10pt]{beamer}

\usetheme{metropolis}
\metroset{numbering=fraction}
\metroset{block=fill}

\usepackage{headings/kmath}
% Tikz
\usetikzlibrary{calc}
\usetikzlibrary{mindmap,trees,shapes,arrows,backgrounds,topaths}
\usetikzlibrary{decorations.pathmorphing, shapes.geometric}

% Text
\usepackage{enumitem}
\usepackage{ulem}
\usepackage{pifont}

% Maths
\usepackage{amsmath}
\usepackage{amsfonts}
\usepackage{amsthm}
\usepackage{amsopn}

% Plots
\usepackage{pgfplots}
\usepgfplotslibrary{groupplots}

% Tables
\usepackage{booktabs}
\usepackage{array}
\newcolumntype{L}{>$l<$}
\arraycolsep=1.4pt
\setlength{\tabcolsep}{3pt}

% Objective values and functions
\newcommand{\pobj}{p}
\newcommand{\robj}{r}
\newcommand{\dobj}{d}

% Variables
\newcommand{\pvletter}{x}
\newcommand{\dvletter}{u}
\newcommand{\bvletter}{z}
\newcommand{\pv}{\mathbf{\pvletter}}
\newcommand{\dv}{\mathbf{\dvletter}}
\newcommand{\bv}{\mathbf{\bvletter}}
\newcommand{\pvi}[1]{\pvletter_{#1}}
\newcommand{\dvi}[1]{\dvletter_{#1}}
\newcommand{\bvi}[1]{\bvletter_{#1}}

% Problem data
\newcommand{\pdim}{n}
\newcommand{\ddim}{m}
\newcommand{\dic}{\mathbf{A}}
\newcommand{\obs}{\mathbf{y}}
\newcommand{\reg}{\lambda}
\newcommand{\lfunc}{f}
\newcommand{\pfunc}{h}
\newcommand{\rfunc}{g}

% Indices
\newcommand{\idxentry}{i}

% BnB
\newcommand{\setidx}{\mathcal{S}}
\newcommand{\setzero}{\setidx_0}
\newcommand{\setone}{\setidx_1}
\newcommand{\setnone}{\setidx_\bullet}
\newcommand{\nodeSymb}{\nu}
\newcommand{\node}[1]{#1^{\nodeSymb}}

% Math operators
\DeclareMathOperator{\dom}{dom}
\DeclareMathOperator{\interior}{int}

% Math misc
\newcommand{\1}{\mathbf{1}}
\newcommand{\0}{\mathbf{0}}
\newcommand{\abs}[1]{|#1|}
\newcommand{\biconj}[1]{#1^{**}}
\newcommand{\conj}[1]{#1^{*}}
\newcommand{\icvx}{\mathbf{I}}
\newcommand{\intervint}[2]{[#1,#2]}
\newcommand{\iter}[2]{#1^{(#2)}}
\newcommand{\norm}[2]{\|#1\|_#2}
\newcommand{\opt}[1]{#1^{\star}}
\newcommand{\separable}[2]{#1_{#2}}
\newcommand{\subdiff}{\partial}
\newcommand{\transp}[1]{#1^{\mathrm{T}}}


\setlength\fboxsep{5pt}
\setlength\fboxrule{1.5pt}


\begin{document}

\begin{frame}[plain,noframenumbering]
  \begin{center}
    {\large\textbf{Branch-and-Bound Algorithms \\ for $\boldsymbol{\ell}_{\boldsymbol{0}}$-Regularized Problems}}
  \end{center}
  \begin{center}
    Théo Guyard \\
    {\small{Inria Rennes - Insa Rennes}} \\
    {\small{27 Novembre 2024}}
    \rule{\textwidth}{1pt}
    {
      \small
      \begin{equation*}
          \begin{array}{L@{}L@{\quad}L@{\hfill}}
              \textbf{Composition du Jury} \\
              Jér\^{o}me Malick & DR CNRS, Université de Grenoble & Rapporteur \\
              Nicolas Gillis & Full Professor, Université de Mons & Rapporteur \\
              Joseph Salmon & Professeur, Université de Montpellier & Examinateur \\
              Amélie Lambert & Ma\^{i}tresse de Conférences, CNAM & Examinatrice \\
              Clément Elvira & Ma\^{i}tre de Conférences, CentraleSupélec & Examinateur \\
              Cédric Herzet & Ma\^{i}tre de Conférences, ENSAI & Directeur \\
          \end{array}
      \end{equation*}
    }
  \end{center}
\end{frame}

\begin{frame}{Intro}
  Intro pour mener au problème
\end{frame}

\begin{frame}{Intro}
  Intro pour mener au problème
\end{frame}

\begin{frame}{Problématiques}
  \begin{itemize}
    \item Problème à résoudre
    \item Diffcultés avant: résolution d'instances relachés: graphique pour montrer que ça marche pas
    \item Nouvelle approche: optim discrète
    \item Algorithms SOTA: BnB
    
  \end{itemize}
\end{frame}

\begin{frame}{BnB}
  \begin{itemize}
    \item Description du BnB
    \item Création des régions
    \item Test de pruning
    \item Difficultés: pas génériques, tjr manque de rapidité
    \item Deux axes de recherche
    \item Spotligh results
  \end{itemize}
\end{frame}

\section{Axe 1: Généralisation}

\begin{frame}{Relaxation Convexe}
  \begin{itemize}
    \item Problème au noeud reformulation
    \item Identification de non-convexité
    \item Bi-conjuguée
    \item Graphique
  \end{itemize}
\end{frame}

\begin{frame}{Charaterizatino}
  \begin{itemize}
    \item Spotlight result: caractérisation de la bi-conjuguée
    \item Aussi prox+sous-diff: algos génériques d'optim
  \end{itemize}
\end{frame}

\section{Axe 2: Réduction du nombre de noeuds}

\begin{frame}{Noeuds de Complexité}
  \begin{itemize}
    \item Nombre de régions explorées
    \item Coût de résolution des relaxations
  \end{itemize}
\end{frame}

\begin{frame}{Bornes Duales}
  \begin{itemize}
    \item Problème Dual
    \item Bornes duales valides: pas bseoin de résoudre à optimalité
    \item Graphique
  \end{itemize}
\end{frame}

\begin{frame}{Charaterizatino}
  \begin{itemize}
    \item Spotlight result: caractérisation de la conjuguée
    \item Aussi prox+sous-diff: algos génériques d'optim
  \end{itemize}
\end{frame}

\begin{frame}{Lien dual}
  \begin{itemize}
    \item Spotligh result: Lien objectif dual entre deux régions
    \item Elaguage pas cher: terme commun
  \end{itemize}
\end{frame}

\begin{frame}{Effet sur le BnB}
  \begin{itemize}
    \item Graphique
  \end{itemize}
\end{frame}

\begin{frame}{Lien dual avec bornes}
  \begin{itemize}
    \item Spotligh result: Lien objectif dual entre deux régions
    \item Peeling
  \end{itemize}
\end{frame}

\begin{frame}{Effet sur le BnB}
  \begin{itemize}
    \item Graphique
  \end{itemize}
\end{frame}

\section{Axe 3: Accélération du traitement des noeuds}

\begin{frame}{Convex}
  \begin{itemize}
    \item Problèlme convexe
    \item Algos typiques de résolution
    \item Méthodes d'accélération
  \end{itemize}
\end{frame}

\begin{frame}{Tests de screening}
  \begin{itemize}
    \item Dual
    \item Test de screening
    \item safe region
    \item Test de safe screening
    \item Spotlight: test de relaxing
  \end{itemize}
\end{frame}

\begin{frame}{Identification complète}
  \begin{itemize}
    \item Spotlight: strong critical et stict convexe
    \item Safe region with sifficiently small radius
  \end{itemize}
\end{frame}

\begin{frame}{Mixed-order}
  \begin{itemize}
    \item Framework
    \item Reformulation
    \item Accélération
  \end{itemize}
\end{frame}

\section{Applications}

\begin{frame}{blabla}
  blabla
\end{frame}

\section{Conclusions et Perspectives}

\begin{frame}
  \begin{itemize}
    \item Résumé des problématiques
    \item Résumé des contributions
    \item 
  \end{itemize}
\end{frame}

\begin{frame}{What's next ?}
  What's next ?
\end{frame}

\begin{frame}[standout]
  Thanks for your attention
\end{frame}

\end{document}
